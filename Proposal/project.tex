\documentclass{article}
\usepackage[nonatbib]{nips_2019}

\usepackage[breaklinks=true,letterpaper=true,colorlinks,citecolor=black,bookmarks=false]{hyperref}

\usepackage{amsthm}
\usepackage{amsmath,amssymb}
\usepackage{enumitem}
\usepackage[normalem]{ulem}

\usepackage[backend=biber,maxbibnames=9,style=numeric,maxcitenames=2,backref=true,uniquelist=false,uniquename=false,sorting=none,defernumbers=true,noerroretextools, url=false,doi=false,isbn=false]{biblatex}
\usepackage{csquotes}
\renewbibmacro{in:}{%
	\ifentrytype{article}{}{%
		\printtext{\bibstring{in}\intitlepunct}}}
\renewbibmacro*{volume+number+eid}{%
	\printfield{volume}%
	%  \setunit*{\adddot}% DELETED
	\setunit*{\addnbspace}% NEW (optional); there's also \addnbthinspace
	\printfield{number}%
	\setunit{\addcomma\space}%
	\printfield{eid}}
%\DeclareFieldFormat[article]{number}{\mkbibparens{#1}} 
\renewbibmacro*{volume+number+eid}{%
	\setunit*{\addcomma\space}% NEW
	\printfield{volume}%
	%  \setunit*{\adddot}% DELETED
	\setunit*{\addcomma\space}% NEW
	\printfield{number}%
	\setunit{\addcomma\space}%
	\printfield{eid}
}
\DeclareFieldFormat[article]{volume}{\bibstring{volume}~#1}% volume of a journal
\DeclareFieldFormat[article]{number}{\bibstring{number}~#1}% number of a journal
\newbibmacro{string+doiurlisbn}[1]{%
  \iffieldundef{doi}{%
    \iffieldundef{url}{%
      \iffieldundef{isbn}{%
        \iffieldundef{issn}{%
          #1%
        }{%
          \href{http://books.google.com/books?vid=ISSN\thefield{issn}}{#1}%
        }%
      }{%
        \href{http://books.google.com/books?vid=ISBN\thefield{isbn}}{#1}%
      }%
    }{%
      \href{\thefield{url}}{#1}%
    }%
  }{%
    \href{http://dx.doi.org/\thefield{doi}}{#1}%
  }%
}
\DeclareFieldFormat{title}{\usebibmacro{string+doiurlisbn}{\mkbibemph{#1}}}
\DeclareFieldFormat[article,incollection,unpublished,inproceedings]{title}%
{\usebibmacro{string+doiurlisbn}{\mkbibquote{#1}}}


% change this to your bib file
\addbibresource{project.bib}

% use Times
\usepackage{times}
% For figures
\usepackage{graphicx} % more modern
%\usepackage{epsfig} % less modern
%\usepackage{subfig} 

\graphicspath{{../fig/}}

\usepackage{tikz}
\usepackage{tkz-tab}
\usepackage{caption} 
\usepackage{subcaption} 
\usetikzlibrary{shapes.geometric, arrows}
\tikzstyle{arrow} = [very thick,->,>=stealth]

\usepackage{cleveref}
\usepackage{setspace}
\usepackage{wrapfig}
%\usepackage[ruled]{algorithm}
\usepackage{algpseudocode}
\usepackage[noend,linesnumbered]{algorithm2e}

\usepackage[disable]{todonotes}


\title{Replace with your title}

\author{
	Yao-Liang Yu \\
	School of Computer Science\\
	University of Waterloo\\
	\texttt{yaoliang.yu@uwaterloo.ca} 
}

\begin{document}
\maketitle

\begin{abstract} 
Put here a brief summary of the project: what is it about and what is your main result.  ``This is a cool paper.''
\end{abstract} 

Please summarize all your findings (empirical, algorithmic, theoretical) in a scientific report. Please always give proper citations to prior work or results. Be precise and concise. We expect the report to be less than \textbf{9 pages} (everything included). You may follow the suggested sections below but feel free to rearrange or adapt as you see fit.

\section{Introduction}
Present a brief background and motivation of your project. State the main results and sell your product. 

\section{Related Works}
Discuss relevant literature. Has your problem, or one of similar nature, been considered before? By whom? What are the differences or limitations (if any)? 

\begin{quote}
\textcite{Novikoff62} improved the convergence proof of \textcite{Block62}. See the excellent book \parencite{SS14} for more on machine learning.
\end{quote}

\section{Problem Formulation}
Concisely and precisely (or even better mathematically) formulate your problem and present the technical challenges (that prior works failed to address).

\section{Main Results}
new algorithm? new analysis? new application?

\section{Experiments}
optional.

\section{Conclusion}


\newpage

\section*{Acknowledgement}
Thank people who have helped or influenced you in this project.

\printbibliography[title={References}]

\end{document}